\documentclass[a4paper,12pt]{report}
\begin{document}
\title{\textbf{A SURVEY ON MOBILE PHONES IN THE WORLD}}
\author{\textbf{SSEMPALA RAYMOND:
		        15/U/12851/EVE}}
\date{\textit{\textbf{7/04/2017}}}
\maketitle
\section{Summary}
{\LARGE The aim of this report was to investigate mobile phones in the world and how they have affected the world in both positive and negative ways.  The survey was conducted and the results show that mobile phones have a great impact on the world and daily lives of people. A sample of the people was conducted on some of the university students.}

\section{Introduction}
{\LARGE  There has been a massive increase in the use of mobile phones over the past years and there is every indication that this will continue. The first hand held mobile phone was demonstrated by John F. Mitchell and Martine Cooper of Motorola in 1973. In 1983, the DynaTac 8000x was first commercially available hand held mobile phone. From 1983 to 2014, worldwide mobile phone subscriptions grew to over 7billion, penetrating 100% of the globally population. With the emerging of smartphones, Samsung, Apple and Huawei were the top manufactures in 2016 with smartphone sales of 78%. These smartphones act like minicomputer with operating system like android, IOS and Blackberry operating system and software like applications for example WhatsApp, Facebook, Instagram etc., camera, messages etc.
According to the university students mobile phones are part of their daily lives. They have greatly positively helped them like in times of need for example in trouble, calling their loved ones etc. , doing their course works, now mobile phones have mobile money services which act like a mini bank of people. With examples like Samsung, Apple, Huawei, Oppo, Xiaomi, Tecno, Nokia etc., these are some of the mobile phones emerging over the years. Mobile phones also have risks like risk of cancer with radiations emitted due to over holding of the phone, they are time consuming (mostly affected by teenagers).
However, nowadays mobile phones use Subscriber Identity Module abbreviated as SIM card. The first SIM card was made in 1991 by Munich smart card maker Giesecke and Devrient. According to University students, mobile phones are mostly used for calling, texting and surfing over the Internet.
}
\section{Conclusion}
{\LARGE  Mobile phones have a great impact on the population of this world and as technology advances each day, mobile phones will continue to have increase in technology advancement due to an increased competition between the different brands but most University students find mobile phones a necessity in their daily live and also research shows it that the rest of the world also can live without a mobile device.}
\pagenumbering{roman}
\end{document}